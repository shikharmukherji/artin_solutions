\documentclass{article}
\usepackage[utf8]{inputenc}
\usepackage{amsfonts}
\usepackage{amsmath}
\usepackage{amssymb}
\usepackage{mathdots}

\title{Chapter 1}
\author{Shikhar Mukherji}
\date{June 2021}

\setlength{\parindent}{0pt}
\begin{document}

\maketitle

\section*{Section 1: The Basic Operations}

\subsection*{Exercise 1.1}

$a_{21}=2$; $a_{23}=8$.

\subsection*{Exercise 1.2}

\[
    \begin{bmatrix}
    1 & 2 & 3\\
    3 & 3 & 1
    \end{bmatrix}
    \begin{bmatrix}
    -8 & -4\\
    9 & 5\\
    -3 & -2
    \end{bmatrix}
    =
    \begin{bmatrix}
    1 & 0\\
    0 & 1
    \end{bmatrix}.
\]

\[
    \begin{bmatrix}
    1 & 4\\
    1 & 2
    \end{bmatrix}
    \begin{bmatrix}
    6 & -4\\
    3 & 2\\
    \end{bmatrix}
    =
    \begin{bmatrix}
    18 & 4\\
    12 & 0
    \end{bmatrix}.
\]

\subsection*{Exercise 1.3}

\[
    AB
    =
    \begin{bmatrix}
    a_1 b_1 + a_2 b_2 + \dots + a_n b_n
    \end{bmatrix}.
\]

\[
    BA
    =
    \begin{bmatrix}
    a_1 b_1 & a_2 b_1 & a_3 b_1 & \dots & a_n b_1\\
    a_1 b_2 & a_2 b_2 & a_3 b_2 & \dots & a_n b_2\\
    \vdots \\
    a_1 b_n & a_2 b_n & a_3 b_n & \dots & a_n b_n\\
    \end{bmatrix}.
\]

\subsection*{Exercise 1.4}

\subsection*{Exercise 1.5}

The $(i,j)^{\text{th}}$ entry of the matrix $AB$ is

$$
\sum_{k=1}^m A_{i,k} B_{k,j},
$$

which requires $m$ multiplications to compute. $AB$ has $ln$  entries, so the total number of multiplications is $lnm$.\\


Since $AB$ is an $l\times n$ matrix, the above tells us that the product $(AB)C$ takes $lnp$ multiplications. Similarly, since $BC$ is an $m\times p$ matrix, the product $A(BC)$ takes $lmp$ multiplications. Thus, if $m>n$ we compute the product in the order $(AB)C$, and if $m<n$ we compute the product in the order $A(BC)$ (if they are equal the order does not matter).

\subsection*{Exercise 1.6}

\[
    \begin{bmatrix}
    1 & a\\
    0 & 1
    \end{bmatrix}
    \begin{bmatrix}
    1 & b\\
    0 & 1\\
    \end{bmatrix}
    =
    \begin{bmatrix}
    1 & a+b\\
    0 & 1
    \end{bmatrix}.
\]

Setting $b=a$ in the above, it is clear that

\[
    \begin{bmatrix}
    1 & a\\
    0 & 1\\
    \end{bmatrix}^n
    =
    \begin{bmatrix}
    1 & na\\
    0 & 1
    \end{bmatrix}.
\]

\subsection*{Exercise 1.7}

We claim that,

\[
    \begin{bmatrix}
    1 & 1 & 1\\
    0 & 1 & 1\\
    0 & 0 & 1\\
    \end{bmatrix}^n
    =
    \begin{bmatrix}
    1 & n & \dfrac{n(n+1)}{2}\\
    0 & 1 & n\\
    0 & 0 & 1\\
    \end{bmatrix}.
\]

Clearly the case when $n=1$ is true. Suppose that the above holds for some integer $k\geq 1$. Then,

\begin{gather*}
    \begin{bmatrix}
    1 & 1 & 1\\
    0 & 1 & 1\\
    0 & 0 & 1\\
    \end{bmatrix}^{k+1}
    =
    \begin{bmatrix}
    1 & 1 & 1\\
    0 & 1 & 1\\
    0 & 0 & 1\\
    \end{bmatrix}^k
    \begin{bmatrix}
    1 & 1 & 1\\
    0 & 1 & 1\\
    0 & 0 & 1\\
    \end{bmatrix}\\ \\
    =
    \begin{bmatrix}
    1 & k & \dfrac{k(k+1)}{2}\\
    0 & 1 & k\\
    0 & 0 & 1\\
    \end{bmatrix}
    \begin{bmatrix}
    1 & 1 & 1\\
    0 & 1 & 1\\
    0 & 0 & 1\\
    \end{bmatrix}
    =
    \begin{bmatrix}
    1 & k+1 & \dfrac{(k+1)(k+2)}{2}\\
    0 & 1 & k+1\\
    0 & 0 & 1\\
    \end{bmatrix},
\end{gather*}

where the final step can be checked by computing the matrix product. Thus, the claim follows by induction.

\subsection*{Exercise 1.8}



\subsection*{Exercise 1.9}

\subsubsection*{a)}

Let $A,B$ be $n\times n$ matrices, and let $C=A+B$ and $D=A-B$. Then,

\begin{gather*}
(CD)_{i,j} = \sum_{k=1}^n C_{i,k} D_{k,j} = \sum_{k=1}^n (A_{i,k} + B_{i,k}) (A_{k,j} - B_{k,j}) \\
= \sum_{k=1}^n A_{i,k} A_{k,j} - A_{i,k} B_{k,j} + B_{i,k} A_{k,j} - B_{i,k}B_{k,j} \\
= A^2_{i,j} - (AB)_{i,j} + (BA)_{i,j} - B^2_{i,j}.
\end{gather*}

Thus, for the $(i,j)^{\text{th}}$ entry of $(A+B)(A-B)$ to be the same as $A^2 - B^2$, we must have $(AB)_{i,j} = (BA)_{i,j}$. In other words, $A$ and $B$ must commute.

\subsubsection*{b)}

We compute the $(i,j)^{\text{th}}$ terms of consecutive powers of $(A+B)$.\\

$$
(A+B)_{ij}=A_{ij}+B_{ij}.
$$

\begin{gather*}
    (A+B)^2_{ij} = \sum_{k=1}^n (A+B)_{ik} (A+B)_{kj} = \sum_{k=1}^n (A_{ik} + B_{ik}) (A_{kj} + B_{kj}) \\
    = \sum_{k=1}^n A_{ik}A_{kj} + A_{ik}B_{kj} + B_{ik}A_{kj} + B_{ik}B_{kj} \\
    = A^2_{ij} + (AB)_{ij} + (BA)_{ij} + B^2_{ij}.
\end{gather*}

\begin{gather*}
    (A+B)^3_{ij} = \sum_{k=1}^n (A+B)^2_{ik} (A+B)_{kj}\\
    = \sum_{k=1}^n (A^2_{ik} + (AB)_{ik} + (BA)_{ik} + B^2_{ik})(A_{kj} + B_{kj}) \\
    = \sum_{k=1}^n A^2_{ik}A_{kj} + (AB)_{ik}A_{kj} + (BA)_{ik}A_{kj} + B^2_{ik}A_{kj} + A^2_{ik}B_{kj} + (AB)_{ik}B_{kj} + (BA)_{ik}B_{kj} + B^2_{ik}B_{kj} \\
    = A^3_{ij} + ABA_{ij} + BA^2_{ij} + B^2A_{ij} + A^2B_{ij} + AB^2_{ij} + BAB_{ij} + B^3_{ij}.
\end{gather*}\\

Thus, $(A+B)^3 = A^3 + ABA + BA^2 + B^2A + A^2B + AB^2 + BAB + B^3$.

\subsection*{Exercise 1.10}
\subsection*{Exercise 1.11}
\subsection*{Exercise 1.12}
\subsection*{Exercise 1.13}
\subsection*{Exercise 1.14}
\subsection*{Exercise 1.15}


\section*{Section 2: Row Reduction}

\subsection*{Exercise 2.1}
\subsection*{Exercise 2.2}
\subsection*{Exercise 2.3}
\subsection*{Exercise 2.4}

Let $x_2=a$, $x_3=b$, $x_4=c$. Then the set of solutions is $(x_1,x_2,x_3,x_4)=(3-a-2b+c,a,b,c)$ for arbitrary values of $a,b$ and $c$.

\subsection*{Exercise 2.5}
\subsection*{Exercise 2.6}
\subsection*{Exercise 2.7}
\subsection*{Exercise 2.8}
\subsection*{Exercise 2.9}

\subsubsection*{a)}
\subsubsection*{b)}

\subsection*{Exercise 2.10}

\section*{Section 3: The Matrix Transpose}

\subsection*{Exercise 3.1}

Using the rules in $(1.3.1)$, $(BB^t)^t = (B^t)^tB^t = BB^t$ and $(B+B^t)^t = B^t + (B^t)^t = B^t + B = B + B^t$.\\

It suffices to show that $A^t(A^{-1})^t=(A^{-1})^tA^t=I$. But, $A^t(A^{-1})^t=(A^{-1}A)^t = I^t = I$, and $(A^{-1})^tA^t = (AA^{-1})^t = I^t = I$.

\subsection*{Exercise 3.2}

If $A$ and $B$ commute, then $(AB)^t=B^tA^t=BA=AB$, so the product $AB$ is symmetric. Conversely, if the product $AB$ is symmetric, then $AB=(AB)^t=B^tA^t=BA$, so $A$ and $B$ commute.

\subsection*{Exercise 3.3}
\subsection*{Exercise 3.4}

\section*{Section 4: Determinants}

\subsection*{Exercise 4.1}
\subsection*{Exercise 4.2}
\subsection*{Exercise 4.3}
\subsection*{Exercise 4.4}

Since each of the $n$ rows of $A$ is multiplied by $-1$ to obtain $-A$, Theorem $1.4.10\text{ (d)}$ tells us that $\text{det}(-A)=(-1)^n\text{det(A)}$.

\subsection*{Exercise 4.5}
\subsection*{Exercise 4.6}

\section*{Section 5: Permutation Matrices}

\subsection*{Exercise 5.1}
\subsection*{Exercise 5.2}

\subsubsection*{a)}

\[
    P=
    \begin{bmatrix}
    0 & 1 & 0 & 0\\
    0 & 0 & 0 & 1\\
    1 & 0 & 0 & 0\\
    0 & 0 & 1 & 0
    \end{bmatrix}
\]

\subsubsection*{b)}



\subsubsection*{c)}

$\text{sign}(p)=\text{det}(P)=-1$.

\subsection*{Exercise 5.3}
\subsection*{Exercise 5.4}

\[
    P=
    \sum_{i=1}^n E_{p(i), i} = \sum_{i=1}^n E_{n-i+1, i} =
    \begin{bmatrix}
    &&&&&&1\\
    &&&&&1&\\
    &&&&1&&\\
    &&&\iddots &&&\\
    &&1&&&&\\
    &1&&&&\\
    1&&&&&\\
    \end{bmatrix}
\]

For even $n$, the cycle decomposition of $p$ is $(1\text{ }n)(2\text{ }n-1)(3\text{ }n-2)\dots (\frac{n}{2}\text{ }\frac{n}{2}+1)$. For odd $n$, it is $(1\text{ }n)(2\text{ }n-1)(3\text{ }n-2)\dots (\frac{n-1}{2}\text{ }\frac{n+3}{2})$.\\

It takes $\lfloor\frac{n}{2}\rfloor$ row interchanges to row reduce $P$ to the identity, and so $\text{sign}(p)=(-1)^{\lfloor\frac{n}{2}\rfloor}$.

\subsection*{Exercise 5.5}

\section*{Section 6: Other Formulas For The Determinant}

\subsection*{Exercise 6.1}

\subsubsection*{a)}
\subsubsection*{b)}
\subsubsection*{c)}

\subsection*{Exercise 6.2}

\section*{Miscellaneous Problems}

\subsection*{Exercise M.1}
\subsection*{Exercise M.2}
\subsection*{Exercise M.3}
\subsection*{Exercise M.4}
\subsection*{Exercise M.5}
\subsection*{Exercise M.6}
\subsection*{Exercise M.7}
\subsection*{Exercise M.8}
\subsection*{Exercise M.9}
\subsection*{Exercise M.10}
\subsection*{Exercise M.11}

\end{document}